\documentclass[12pt,a4paper]{article}
\usepackage[utf8]{inputenc}
\usepackage[T1]{fontenc}
\usepackage[french]{babel}
\usepackage{graphicx}
\usepackage{amsmath}
\usepackage{geometry}
\usepackage{hyperref}
\usepackage{lmodern}
\usepackage{titlesec}
\usepackage{fancyhdr}

\setlength{\headheight}{14.5pt}

\geometry{margin=2.5cm}
\pagestyle{fancy}
\fancyhf{}
\rhead{\thepage}
\lhead{Initiation à la Recherche}

\title{Création de PNJ basés sur l'expérience utilisateur via l'apprentissage machine}
\author{Noa JELSCH, Yannick LOIS, Benoît MULLER\\CESI École d'Ingénieurs}
\date{11 Juillet 2025}

\begin{document}

\maketitle
\tableofcontents
\newpage

\section{Introduction}
\begin{itemize}
    \item Contexte général du sujet
    \item Problématique et justification scientifique
    \item Objectifs de la recherche
\end{itemize}

\section{État de l'art / Revue de littérature}
\subsection{IA et comportements humains dans les jeux}
\subsection{Méthodes de collecte de données comportementales}
\subsection{Apprentissage machine pour PNJ}
\subsection{Cas d'étude existants}
\subsection{Outils et frameworks disponibles}

\section{Méthodologie}
\begin{itemize}
    \item Choix du jeu et du type de données à collecter
    \item Prétraitement des données et sélection des features
    \item Modèle d'apprentissage retenu (supervisé ou RL)
    \item Justification des choix techniques
\end{itemize}

\section{Réalisation et expérimentations}
\begin{itemize}
    \item Collecte et traitement des données
    \item Entraînement du modèle
    \item Évaluation des performances
    \item Analyse des résultats
\end{itemize}

\section{Synthèse et discussion}
\begin{itemize}
    \item Résumé des apports
    \item Limites de l'approche
    \item Perspectives d'amélioration
    \item Réflexion personnelle et lien avec le projet professionnel
\end{itemize}

\section{Conclusion}

\section*{Annexes}
\addcontentsline{toc}{section}{Annexes}

\section*{Références bibliographiques}
\addcontentsline{toc}{section}{Références bibliographiques}

\end{document}